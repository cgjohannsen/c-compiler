\documentclass{article}

\usepackage{graphicx}
\usepackage[margin=1in]{geometry}

\linespread{1.6}

\title{Basic C Compiler Developers Guide}
\author{Chris Johannsen}
\date{\today}

\begin{document}

\maketitle

\section{Introduction}

This document provides information on my C compiler. We only implement a subset of the language which will be defined as the document is developed.

\section{Building}

To build the both the documentation and compiler use the Makefile provided in the \verb|compiler/| directory and run:

\begin{verbatim}
    make
\end{verbatim}

\noindent To build just the documentation run:

\begin{verbatim}
    make developers
\end{verbatim}

\noindent To build just the compiler run:

\begin{verbatim}
    make mycc
\end{verbatim}

\noindent To clean all output files run:

\begin{verbatim}
    make clean
\end{verbatim}

\section{Usage}

\begin{verbatim}
Usage: mycc -mode [-o outfile] [-h] infile
    mode        integer from 0-5 specifying mode to run
    outfile     file to write output to instead od stdout
    infile      file to read input from
\end{verbatim}

\section{Modes}

In this section we describe the features of the various modes the compiler can be run in.

\subsection{Mode 0}

\subsubsection{Features}

\begin{itemize}
    \item Usage and version information can be printed
    \item Output can be printed to a specified file
\end{itemize}

\subsubsection{Implementation Details}

We use \verb|getopt| to parse input arguments. If no mode is specified, the program prints usage information and exits. Ff mode 0 is specified, the program prints version information. Otherwise the program prints nothing if any other mode is spceified. 

When run in this mode we ignore the input file; whether one is specified or not.

\end{document}