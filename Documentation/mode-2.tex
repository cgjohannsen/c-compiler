\subsection{Mode 2}
\label{sec:mode-2}

This mode corresponds to the preprocessor.

\subsubsection{Features}

\begin{itemize}
    \item Defining and undefining of macros
    \item Inclusion of other files
    \item Basic logic in the form of ifdefs
\end{itemize}

\subsubsection{Implementation Details}

\textbf{Relevant Files:}

\begin{itemize}
    \item \verb|parse/lexer.c|: All features are expansions on the lexer
\end{itemize}

\noindent \textbf{Relevant Data Structures:}

\begin{itemize}
    \item \verb|include_t|: A double linked list to keep track of all file
    inclusions (for cycle detection)
    \item \verb|macro_t|: A linked list of currently defined macros
    \item \verb|ifdef_t|: We use a stack to keep track of the current 
\end{itemize}

\noindent \textbf{Working Description:} The preprocessor can recognize
preprocessor directives when it comes across a `\#' symbol. When it does, it
dispatches to the corresponding function of the directive provided. 

In the case of a file inclusion, the lexer will check if the file is already in
the list of included files and if so outputs an error as this is a cycle.
Otherwise, it generates a new lexer that then generates tokens in the new file.
While storing the current state of the original file.

For macro definitions and expansions, we use a linked list. We add and subtract
from this list as macros are defined/undefined. Macro redefinition is not
allowed. To expand a macro, the lexer's current character changes to the
location of the macro text while the macro stores the location of the original
expansion that the lexer goe back to after it is done.

For ifdefs, the lexer uses a stack to keep track of whether the condition of the
current ifdef was true. If so, it skips the next else directive. 
