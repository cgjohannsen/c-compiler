\subsection{Mode 3}
\label{sec:mode-3}

This mode corresponds to the parser and reports if the input file has valid syntax.

\subsubsection{Features}

\begin{itemize}
    \item Parses the input file and reports if file has valid syntax
    \item Reports first syntax error in the file (at a minimum)
    \item Supports variable initialization
    \item Supports structs
\end{itemize}

\subsubsection{Implementation Details}

\textbf{Relevant Files:}

\begin{itemize}
    \item \verb|parse/parse.c|: Implements the parser. Uses the lexer and token
    structures as seen in Sec \ref{sec:mode-1}.
\end{itemize}

\noindent \textbf{Relevant Data Structures:}

\begin{itemize}
    \item \verb|parser|: Stores data relevant to the file being parsed. Relevant
    data includes the lexer (as in Sec~\ref{sec:mode-1}), current and next
    tokens in the input stream, and the abstract syntax tree (to be
    implemented).
\end{itemize}

\noindent \textbf{Working Description:} The parser is a top-down
recursive-descent parser. As such, we attempt to implement the grammar defined
in the \verb|grammar.g| file as close as possible (with some exceptions due to
implementation ease). In general, most non-terminals have a function associated
with them that we use to parse. The grammar lends itself to only needing
one-lookahead at most (which is seldom used).

The parser essentially works in a while loop, attempting to match the input
against each of the top-level rules (i.e., a variable declaration, function
prototype, or function definition). If a token is found that does not match the
expected next token, the parser reports an error and returns from the respective
function. The utlity of being recursive descent meamns that exiting in case of
error requires only a simple call to \verb|return|. 

If the parser makes it through the entire token input stream without an error,
it reports back to the user that the input file is syntactically correct.

The abstract syntax tree is not yet implemented, but will be for future modes.