\subsection{Mode 4}
\label{sec:mode-4}

This mode corresponds to the type checker and will output each of the 

\subsubsection{Features}

\begin{itemize}
    \item Parser now returns an Abstract Syntax Tree (AST) corresponding to the input program.
    \item Global and local variable, type, and function declarations are printed.
\end{itemize}

\subsubsection{Implementation Details}

\textbf{Relevant Files:}

\begin{itemize}
    \item \verb|parse/parse.c|: This file was revamped from mode 3 so that it now builds an AST as well as just checking for syntax errors.
    \item \verb|parse/ast.c|: The data structure corresponding to the AST.
    \item \verb|parse/typecheck.c|: This file holds all the functionality for performing type checking of the input file.
\end{itemize}

\noindent \textbf{Relevant Data Structures:}

\begin{itemize}
    \item \verb|astnode_t|: This is a tree structure that serves as the AST of the program. Functionally, the AST is a binary tree where the left child is the child of the current node and the right child is the next sibling. 
\end{itemize}

\noindent \textbf{Working Description:} 