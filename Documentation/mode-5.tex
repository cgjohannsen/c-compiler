\subsection{Mode 5}
\label{sec:mode-5}

This mode corresponds to the code generator and will output the corresponding
Java bytecode associated with the C input file.

\subsubsection{Features}

\begin{itemize}
    \item Outputs Java bytecode for input file. Works for functions, expressions
    (non-logical), and arrays. 
\end{itemize}

\subsubsection{Implementation Details}

\textbf{Relevant Files:}

\begin{itemize}
    \item \verb|gen.c|: This file includes code for generating Java bytecode.
    \item \verb|instruction.c|: The data structure corresponding to Java
    bytecode instructions.
\end{itemize}

\noindent \textbf{Relevant Data Structures:}

\begin{itemize}
    \item \verb|instruction_t|: This holds the data relevant for an instruction
    including the text and type.
    \item \verb|instruction_list_t|: This stores a list of instructions so that
    the proper annotations can be printed for each function before the
    instructions are in the output. 
\end{itemize}

\noindent \textbf{Working Description:} The code generator works by traversing
the AST and generating code based on its structure. Assuming the parser
functions properly, the output will be valid Java bytecode.

For functions, the generator outputs the correct signature then generates the
corresponding body of the function, storing the list of instructions
temporarily. While doing this, it keeps track of the max stack size and the
number of local variables. This way the relevant information is printed at the
top of the function as needed, then the list of instructions for the body of the
function is printed.