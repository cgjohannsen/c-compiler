\subsection{Mode 6}
\label{sec:mode-6}

This mode corresponds to the code generator and will output the corresponding
Java bytecode associated with the C input file.

\subsubsection{Features}

\begin{itemize}
    \item Outputs Java bytecode for input file. Works for functions,
    expressions, arrays, and other program structures (e.g. loops).
\end{itemize}

\subsubsection{Implementation Details}

\textbf{Relevant Files:}

\begin{itemize}
    \item \verb|gen.c|: This file includes code for generating Java bytecode.
    \item \verb|instruction.c|: The data structure corresponding to Java
    bytecode instructions.
\end{itemize}

\noindent \textbf{Relevant Data Structures:}

\begin{itemize}
    \item \verb|instruction_t|: This holds the data relevant for an instruction
    including the text and type.
    \item \verb|instruction_list_t|: This stores a list of instructions so that
    the proper annotations can be printed for each function before the
    instructions are in the output. 
\end{itemize}

\noindent \textbf{Working Description:} The code generator functions the same in
this mode as in Mode 5~\ref{sec:mode-5}. It extends this mode to allow for for,
while, do-while loops. It does this by dropping labels in the bytecode at
proper locations and providing branching instructions to properly implement each
loop. It provides functionality for if-then-else statements in a similar manner.